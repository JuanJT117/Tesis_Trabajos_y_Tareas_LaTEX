\chapter{Planteamiento del problema}

\section{Modelo de vulnerabilidad hidrológica}

	Actualmente los modelos de susceptibilidad hidrológica se desarrollan a partir de la sumatorias del producto de pesos asignados a propiedades o características identificadas y clasificadas, usualmente agregando asignaciones en un modelo raster, el cual consiste de una estructura matricial, la cual mantiene una distribución igual de pixeles o tamaño de celda, conteniendo una o varias bandas, cuyos valores corresponden a atributos asignados o inferidos a partir de modelos de interpolación.\\
	
	los modelos de vulnerabilidad presentan el inconveniente de poder ser generados con datos des actualizados, por lo que se carece de la capacidad de ser una herramienta que nos permita identificar cambios en el tiempo o realizar predicciones con base en las diferencias de distribución o cambios en el sitio donde se generan estos datos.\\
	
	Entendiendo esta problemática se propone realizar una actualización del modelo a partir de datos actuales de disponibilidad geohidrológica generados empleando los datos de la misión Gravity Recovery and Climate Experiment (GRACE), mediante los cuales es posible predecir los niveles de los acuíferos y de los cuales se 
	